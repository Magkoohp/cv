
%----------------------------------------------------------------------------------------
%	PACKAGES AND OTHER DOCUMENT CONFIGURATIONS
%----------------------------------------------------------------------------------------

\documentclass{resume} % Use the custom resume.cls style

\usepackage[left=0.75in,top=0.6in,right=0.75in,bottom=0.6in]{geometry} % Document margins
\newcommand{\tab}[1]{\hspace{.2667\textwidth}\rlap{#1}}
\newcommand{\itab}[1]{\hspace{0em}\rlap{#1}}
\name{Runchao Han} % Your name
\address{25 Exhibition Walk, Clayton VIC 3800, Australia} % Your address
%\address{123 Pleasant Lane \\ City, State 12345} % Your secondary addess (optional)
\address{runchao.han@monash.edu \\ elvisage941102@gmail.com}

\begin{document}

%----------------------------------------------------------------------------------------
%	EDUCATION SECTION
%----------------------------------------------------------------------------------------

\begin{rSection}{Education}

{\bf Monash University and CSIRO-Data61} \hfill {\em February 2019 - Present} 
\\ Doctor of Philosophy\hfill { Supervisors: Jiangshan Yu, Joseph Liu and Shiping Chen}
\\ Faculty of Information Technology and Distributed Systems Security group\\
%
\\{\bf The University of Manchester} \hfill {\em August 2017 - September 2018} 
\\ MSc Advanced Computer Science\hfill {Supervisor: Christos Kotselidis}
\\ School of Computer Science\hfill { Overall Percentage: 82/100 }\\
%
\\{\bf Beijing University of Posts and Telecommunications} \hfill {\em September 2013 - July 2017} 
\\ BSc E-Commerce Engineering with Law\hfill { Overall Percentage: 83/100 }
%Minor in Linguistics \smallskip \\
%Member of Eta Kappa Nu \\
%Member of Upsilon Pi Epsilon \\


\end{rSection}

% \begin{rSection}{Carrier Objective}
%  To work for an organization which provides me the opportunity to improve my skills and knowledge to grow along with the organization objective.
% \end{rSection}
%--------------------------------------------------------------------------------
%    Projects And Seminars
%-----------------------------------------------------------------------------------------------
\begin{rSection}{Research}


\textbf{On the optionality and fairness of Atomic Swap}
\begin{itemize}
    \item We showed the HTLC-based Atomic Swap equals to an American Call Option without premium.
    \item We evaluated the unfairness of HTLC-based Atomic Swaps using Cox-Ross-Rubinstein model.
    \item We fixed such unfairness of HTLC-based Atomic Swap by paying the premium during the swap.
\end{itemize}

\textbf{Challenging the honest majority assumption of permissionless blockchains}
\begin{itemize}
    \item We experimentally showed that the current incentive mechanism may encourage rational participants to launch 51\% attacks.
    \item We formally modelled 51\% attacks with external mining power as a Markov Decision Process.
    \item We evaluated such 51\% attacks and found that, 51\% attacks are feasible and profitable for most mainstream PoW-based blockchains.
\end{itemize}

\textbf{Analysis and Optimizations of Memory-Hard PoW Algorithms}
\begin{itemize}
    \item We surveyed state-of-the-art Proof-of-Work algorithms deployed in prevalent cryptocurrencies.
    \item We profiled CUDA implemented memory-hard PoW algorithms: Ethash, CryptoNight and Scrypt.
    \item We optimised the Ethash CUDA implementation on Nvidia's GPUs.
\end{itemize}

\end{rSection}




%----------------------------------------------------------------------------------------
%	WORK EXPERIENCE SECTION
%----------------------------------------------------------------------------------------

\begin{rSection}{Work Experience}

\begin{rSubsection}{Bytom Blockchain, Hangzhou, China}{September 2018 - January 2019}{Intern Blockchain Engineer}{Hangzhou, China}
\item Developed and refactored the original mining pool of the Bytom blockchain
\item Developed the first version of the heterogeneous desktop grid computing platform based on the mining pool and the AI cloud platform
\item Research on state-of-the-art technologies to improve the Bytom blockchain (sharding, consensus and P2P) and the cryptocurrency mining pool
\end{rSubsection}

\begin{rSubsection}{CNIC, Chinese Academy of Sciences, Beijing, China}{June 2017 - July 2017}{Intern Researcher}{Beijing, China}
\item Designed the Solidity smart contracts for electronic licenses library based on Ethereum
\item Deployed smart contracts on Docker containers of Geth/TestRPC clients
\item Implemented the front-end client of the DApp based on Web3 and Truffle
\end{rSubsection}

\end{rSection}



%----------------------------------------------------------------------------------------
%	PUBLICATION SECTION
%----------------------------------------------------------------------------------------

\begin{rSection}{Selected publications} 

\begin{itemize}
    \item Runchao Han, Zhimei Sui, Jiangshan Yu, Joseph Liu, Shiping Chen. Challenging the honest majority assumption of permissionless blockchains. In submission.
    \item Runchao Han, Haoyu Lin, Jiangshan Yu. On the optionality and fairness of Atomic Swaps. AFT'19.
    \item Runchao Han, Nikolaos Foutris and Christos Kotselidis. Demystifying Crypto Mining: Performance Analysis and Optimizations of PoW Algorithms. ISPASS'19.
\end{itemize}

\end{rSection}


%----------------------------------------------------------------------------------------
%	TECHNICAL STRENGTHS SECTION
%----------------------------------------------------------------------------------------

\begin{rSection}{Technical Strengths}

\begin{tabular}{ @{} >{\bfseries}l @{\hspace{6ex}} l }
Programming Languages & C++, Go, Rust, Java, Python \\
Blockchains & Bitcoin, Ethereum, Monero \\
Parallel Programming & CUDA, OpenMP, MPI \\
Cryptography & Libsodium, Ring (Rust), golang/crypto\\
\end{tabular}

\end{rSection}

%----------------------------------------------------------------------------------------
%	ADDITIONAL INFO SECTION
%----------------------------------------------------------------------------------------

\begin{rSection}{Additional Information}

\begin{tabular}{ @{} >{\bfseries}l @{\hspace{6ex}} l }
Github & SebastianElvis \\
Blog & SebastianElvis.github.io \\
Wechat & elvisage \\
\end{tabular}

\end{rSection}

\end{document}
