
%----------------------------------------------------------------------------------------
%	PACKAGES AND OTHER DOCUMENT CONFIGURATIONS
%----------------------------------------------------------------------------------------

\documentclass{resume} % Use the custom resume.cls style

\usepackage[left=0.75in,top=0.6in,right=0.75in,bottom=0.6in]{geometry} % Document margins
\usepackage{url}
\usepackage{hyperref}
\hypersetup{
    colorlinks=true,
    linkcolor=blue,
    filecolor=magenta,      
    urlcolor=cyan,
}

\newcommand{\tab}[1]{\hspace{.2667\textwidth}\rlap{#1}}
\newcommand{\itab}[1]{\hspace{0em}\rlap{#1}}
\name{Runchao Han} % Your name
\address{20 Exhibition Walk, Clayton VIC 3800, Australia} % Your address
%\address{123 Pleasant Lane \\ City, State 12345} % Your secondary addess (optional)
\address{runchao.han@monash.edu \\ \url{https://runchao.rocks}}

\begin{document}


\begin{rSection}{Employment}
    \begin{rSubsubsection}{BabylonChain Inc.}{July 2022 - Now}{Senior Research Engineer}{Melbourne, Australia}
    \end{rSubsubsection}

    \begin{rSubsubsection}{Bytom Blockchain}{September 2018 - January 2019}{Intern Blockchain Engineer}{Hangzhou, China}
    \end{rSubsubsection}

    \begin{rSubsubsection}{CNIC, Chinese Academy of Sciences}{June 2017 - July 2017}{Intern Researcher}{Beijing, China}
    \end{rSubsubsection}
\end{rSection}

\begin{rSection}{Education}

    {\bf Monash University and CSIRO's Data61} \hfill {\em February 2019 - September 2022}
    \\ Doctor of Philosophy\hfill { Supervisors: Jiangshan Yu, Joseph Liu and Shiping Chen}
    \\ Department of Software Systems and Cybersecurity\\
    %
    \\{\bf The University of Manchester} \hfill {\em August 2017 - September 2018}
    \\ MSc Advanced Computer Science (with Distinction)\hfill {Supervisor: Christos Kotselidis}
    \\ School of Computer Science\hfill { Overall Percentage: 82/100 }\\
    %
    \\{\bf Beijing University of Posts and Telecommunications} \hfill {\em September 2013 - July 2017}
    \\ BSc E-Commerce Engineering with Law\hfill { Overall Percentage: 83/100 }

\end{rSection}

%----------------------------------------------------------------------------------------
%	PUBLICATION SECTION
%----------------------------------------------------------------------------------------


\begin{rSection}{Publications}

    Full publication list can be found at \href{https://dblp.org/pers/hd/h/Han:Runchao}{DBLP} and \href{http://scholar.google.com/citations?user=xbpDocQAAAAJ&hl=en}{Google Scholar}.
    All of my papers are available online, and most of them are hosted on \href{https://eprint.iacr.org/}{IACR ePrint}.

    \begin{itemize}
        \item[\href{}{HY23}] Fair Delivery of Decentralised Randomness Beacon. Runchao Han, Jiangshan Yu. The 27th International Conference on Financial Cryptography and Data Security \href{https://fc23.ifca.ai/}{(FC'23)}.
        \item[\href{https://eprint.iacr.org/2020/943}{HYZ22}] Analysing and Improving Shard Allocation Protocols for Sharded Blockchains. Runchao Han, Jiangshan Yu, Ren Zhang. The 4th ACM Conference on Advances in Financial Technologies \href{https://aft.acm.org/aft22/index.html}{(AFT'22)}.
        \item[\href{}{HHD+22}] Reputation-based state machine replication. Muhong Huang, Runchao Han, Zhiqiang Du, Yanfang Fu, and Liangxin Liu. The 21th IEEE International Symposium on Network Computing and Applications \href{https://www.nca-ieee.org/2022/index.html}{(NCA'22)}.
        \item[\href{https://ieeexplore.ieee.org/document/9927474/}{NGH+22}] Crystal: Enhancing Blockchain Mining Transparency with Quorum Certificate. Jianyu Niu, Fangyu Gai, Runchao Han, Ren Zhang, Yinqian Zhang, Chen Feng. IEEE Transactions on Dependable and Secure Computing \href{https://ieeexplore.ieee.org/xpl/RecentIssue.jsp?punumber=8858}{(TDSC'22)}
        \item[\href{https://eprint.iacr.org/2020/456.pdf}{LHY20}] General Congestion Attack on HTLC-Based Payment Channel Networks. Zhichun Lu, Runchao Han, Jiangshan Yu. The 3rd International Conference on Blockchain Economics, Security and Protocols \href{https://drops.dagstuhl.de/opus/volltexte/2022/15899/}{(Tokenomics'21)}.
        \item[\href{https://eprint.iacr.org/2019/752}{HSY+20}] Fact and Fiction: Challenging the honest majority assumption of permissionless blockchains. Runchao Han, Zhimei Sui, Jiangshan Yu, Joseph Liu, Shiping Chen. The 16th ACM ASIA Conference on Computer and Communications Security \href{https://dl.acm.org/doi/10.1145/3433210.3453087}{(AsiaCCS'21)}.
        \item[\href{https://github.com/DEX-ware/vrf-mining/blob/master/paper/main.pdf}{HYL20a}] VRF-based mining: Simple Non-outsourceable Cryptocurrency Mining. Runchao Han, Haoyu Lin, Jiangshan Yu. The 4th International Workshop on Cryptocurrencies and Blockchain Technology, in conjunction with the 25th European Symposium on Research in Computer Security \href{https://link.springer.com/chapter/10.1007/978-3-030-66172-4_19}{(CBT@ESORICS'20)}.
        \item[\href{https://eprint.iacr.org/2019/896}{HLY19}] On the optionality and fairness of Atomic Swaps. Runchao Han, Haoyu Lin, Jiangshan Yu. The 1st ACM Conference on Advances in Financial Technologies \href{https://dl.acm.org/doi/10.1145/3318041.3355460}{(AFT'19)}.
        \item[\href{https://www.research.manchester.ac.uk/portal/files/85753741/paper.pdf}{HFK19}] Demystifying Crypto Mining: Performance Analysis and Optimizations of PoW Algorithms. Runchao Han, Nikolaos Foutris, Christos Kotselidis. IEEE International Symposium on Performance Analysis of Systems and Software \href{https://ieeexplore.ieee.org/document/8695663}{(ISPASS'19, best paper nominee)}.
        \item[\href{https://gramoli.redbellyblockchain.io/web/doc/pubs/IoT2019.pdf}{HSGX19}] On the performance of distributed ledgers for internet of things. Runchao Han, Gary Shapiro, Vincent Gramoli, Xiwei Xu. Internet of Things; Engineering Cyber Physical Human Systems \href{https://www.sciencedirect.com/science/article/abs/pii/S2542660518300416}{(Elsevier IoT'19)}.
        \item[\href{https://www.researchgate.net/profile/Runchao_Han/publication/331227984_Evaluating_CryptoNote-Style_Blockchains_14th_International_Conference_Inscrypt_2018_Fuzhou_China_December_14-17_2018_Revised_Selected_Papers/links/5c747901299bf1268d25a5f5/Evaluating-CryptoNote-Style-Blockchains-14th-International-Conference-Inscrypt-2018-Fuzhou-China-December-14-17-2018-Revised-Selected-Papers.pdf}{HYLZ18}] Evaluating CryptoNote-Style Blockchains. Runchao Han, Jiangshan Yu, Joseph Liu, Peng Zhang. International Conference on Information Security and Cryptology \href{https://link.springer.com/chapter/10.1007/978-3-030-14234-6_2}{(Inscrypt'18)}.
        \item[\href{https://gramoli.redbellyblockchain.io/web/doc/pubs2/blockchain-iot.pdf}{HGX18}] Evaluating Blockchains for IoT. Runchao Han, Vincent Gramoli, Xiwei Xu. The 9th IFIP International Conference on New Technologies, Mobility and Security \href{https://ieeexplore.ieee.org/document/8328736}{(NTMS'18)}.
    \end{itemize}

\end{rSection}


\begin{rSection}{Preprints}
    \item[\href{https://eprint.iacr.org/2020/1033}{HYL+20}] On the Security and Performance of Blockchain Sharding. Runchao Han, Jiangshan Yu, Haoyu Lin, Shiping Chen, Paulo Esteves-Veríssimo. \textbf{In submission}.
    \item[\href{https://eprint.iacr.org/2020/1033}{HYL20}] \textsc{RandChain}: Decentralised Randomness Beacon from Sequential Proof-of-Work. Runchao Han, Jiangshan Yu, Haoyu Lin. \textbf{In submission}.
\end{rSection}

\begin{rSection}{Talks}
    \begin{itemize}
        \item Analysing and Improving Shard Allocation Protocols for Sharded Blockchains. AFT conference talk. September, 2022.
        \item Fact and Fiction: Challenging the honest majority assumption of permissionless blockchains. AsiaCCS conference talk. June, 2021.
        \item RandChain: Decentralised Randomness Beacon from Sequential Proof-of-Work.
              \begin{itemize}
                  \item Seminar at Nervos Foundation. June, 2022.
                  \item Seminar at Decrypto. November, 2020.
              \end{itemize}
        \item VRF-Based Mining: Simple Non-Outsourceable Cryptocurrency Mining. CBT@ESORICS workshop. September, 2020.
        \item Demystifying Crypto-Mining: Analysis and Optimizations of Memory-Hard PoW Algorithms. Seminar at Huawei Noah's Ark Lab. July, 2020.
        \item On the optionality and fairness of Atomic Swaps.
              \begin{itemize}
                  \item Peep an EIP \#23: EIP-2266. February, 2021.
                  \item AFT conference at Zurich. October, 2019.
              \end{itemize}
    \end{itemize}
\end{rSection}


\begin{rSection}{Teaching}
    \begin{itemize}
        \item Teaching associate for FIT 5214 Blockchain, Monash University. 2019 Fall.
    \end{itemize}
\end{rSection}


\begin{rSection}{Professional services}
    \subsection*{(External) reviewer}
    \begin{itemize}
        \item 2022: EuroS\&P, IEEE Blockchain, Journal of Network and Systems Management, ACM DLT Journal, IEEE Transaction of Service Computing
        \item 2021: DSN, ICBC, IEEE Blockchain, NSS, MSN, TrustCom, IEEE Trans.\ of Service Computing
        \item 2020: AFT, ICDCS, SRDS, ACNS, AsiaCCS, TrustCom, ACISP, ICBC, TDSC, The Computer Journal, IEEE IoT Journal, IEEE Software Journal, IEEE Trans.\ of Service Computing
        \item 2019: Indocrypt, TrustCom, Future Generation Computing System
    \end{itemize}
\end{rSection}

\begin{rSection}{Additional Information}

    \begin{tabular}{ @{} >{\bfseries}l @{\hspace{6ex}} l }
        Membership & ACM (student), IEEE (student), USENIX, IACR \\
        LinkedIn   & runchao-han                                 \\
        Github     & SebastianElvis                              \\
        Wechat     & elvisage                                    \\
        References & Available upon request
    \end{tabular}

\end{rSection}

\end{document}
