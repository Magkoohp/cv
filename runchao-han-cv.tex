
%----------------------------------------------------------------------------------------
%	PACKAGES AND OTHER DOCUMENT CONFIGURATIONS
%----------------------------------------------------------------------------------------

\documentclass{resume} % Use the custom resume.cls style

\usepackage[left=0.75in,top=0.6in,right=0.75in,bottom=0.6in]{geometry} % Document margins
\usepackage{url}
\usepackage{hyperref}
\hypersetup{
    colorlinks=true,
    linkcolor=blue,
    filecolor=magenta,      
    urlcolor=cyan,
}

\newcommand{\tab}[1]{\hspace{.2667\textwidth}\rlap{#1}}
\newcommand{\itab}[1]{\hspace{0em}\rlap{#1}}
\name{Runchao Han} % Your name
\address{25 Exhibition Walk, Clayton VIC 3800, Australia} % Your address
%\address{123 Pleasant Lane \\ City, State 12345} % Your secondary addess (optional)
\address{runchao.han@monash.edu \\ \url{https://runchao.rocks}}

\begin{document}

%----------------------------------------------------------------------------------------
%	EDUCATION SECTION
%----------------------------------------------------------------------------------------

\begin{rSection}{Education}

{\bf Monash University and CSIRO-Data61} \hfill {\em February 2019 - Present} 
\\ Doctor of Philosophy\hfill { Supervisors: Jiangshan Yu, Joseph Liu and Shiping Chen}
\\ Faculty of Information Technology and Distributed Systems Security group\\
%
\\{\bf The University of Manchester} \hfill {\em August 2017 - September 2018} 
\\ MSc Advanced Computer Science (with Distinction)\hfill {Supervisor: Christos Kotselidis}
\\ School of Computer Science\hfill { Overall Percentage: 82/100 }\\
%
\\{\bf Beijing University of Posts and Telecommunications} \hfill {\em September 2013 - July 2017} 
\\ BSc E-Commerce Engineering with Law\hfill { Overall Percentage: 83/100 }
%Minor in Linguistics \smallskip \\
%Member of Eta Kappa Nu \\
%Member of Upsilon Pi Epsilon \\


\end{rSection}

% \begin{rSection}{Carrier Objective}
%  To work for an organization which provides me the opportunity to improve my skills and knowledge to grow along with the organization objective.
% \end{rSection}
%--------------------------------------------------------------------------------
%    Projects And Seminars
%-----------------------------------------------------------------------------------------------
\begin{rSection}{Research and impact}

I'm broadly interested in distributed system security.
Currently, I'm focusing on designing secure and scalable Blockchains and decentralised protocols (e.g., Decentralised Randomness Beacon).
My research applies techniques from Cryptography and Distributed Computing.

My research has led to real-world impacts and media coverage. For example,
\begin{itemize}
    \item I invented \textsc{RandChain} [HYL20], a new family of Decentralised Randomness Beacon protocols that are simple, secure and scalable. This research  is featured in \href{https://vdfresearch.org/}{VDF Research}.
    \item My paper analysing \emph{shard allocation} (a key component and a missing abstraction in sharded blockchains) [HYZ20] is selected as \href{https://zkcapital.substack.com/}{"Paper of the Week" (Issue \#68)} by ZK Capital.
    \item I study two overlooked 51\% attacks on PoW-based blockchains [HSY+20]. Three large-scale 51\% attacks on Ethereum Classic (\href{https://news.bitcoin.com/ethereum-classic-suffers-51-attack-again-delisting-risk-amplified}{1},\href{https://decrypt.co/40196/hackers-launch-third-51-attack-on-ethereum-classic-this-month}{2},\href{https://coingeek.com/over-1m-double-spent-in-latest-ethereum-classic-51-attack}{3}) happened within a month are likely to be our analysed attacks.
    \item I identify and formalise an overlooked design flaw of the Atomic Swap protocol [HLY19]. The flaw allows the swap initiator to arbitrage, making the protocol unfair. Our proposed fixes are standardised as an Ethereum Improvement Proposal (\href{https://github.com/ethereum/EIPs/issues/2266}{EIP-2266}). This research is covered by \href{https://cryptonews.com.au/monash-university-researchers-developing-cryptocurrency-transaction-platform}{CryptoNews}.
    \item I conduct the first performance analysis on memory-hard cryptocurrency mining algorithms [HFK19]. This research is covered by \href{https://medium.com/@horizonfintex/blockchain-research-bytes-1-9d023e080765}{Horizon Globex}.
\end{itemize}

\end{rSection}

%----------------------------------------------------------------------------------------
%	PUBLICATION SECTION
%----------------------------------------------------------------------------------------


\begin{rSection}{Selected publications} 

Full publication list can be found at \href{https://dblp.org/pers/hd/h/Han:Runchao}{DBLP} and \href{http://scholar.google.com/citations?user=xbpDocQAAAAJ&hl=en}{Google Scholar}.
All of my papers are available online, and most of them are hosted on IACR ePrint.

\begin{itemize}
    \item[\href{https://eprint.iacr.org/2020/1033}{HYL20}] \textsc{RandChain}: Decentralised Randomness Beacon from Sequential Proof-of-Work. Runchao Han, Jiangshan Yu, Haoyu Lin. \textbf{In submission}.
    \item[\href{https://eprint.iacr.org/2020/943}{HYZ20}] Analysing and Improving Shard Allocation Protocols for Sharded Blockchains. Runchao Han, Jiangshan Yu, Ren Zhang. \textbf{In submission}.
    \item[\href{https://eprint.iacr.org/2019/752}{HSY+20}] Challenging the honest majority assumption of permissionless blockchains. Runchao Han, Zhimei Sui, Jiangshan Yu, Joseph Liu, Shiping Chen.  \textbf{In submission}.
    \item[\href{https://eprint.iacr.org/2019/896}{HLY19}] On the optionality and fairness of Atomic Swaps. Runchao Han, Haoyu Lin, Jiangshan Yu.  ACM Conference on Advances in Financial Technologies. \textbf{AFT'19}.
    \item[\href{https://www.research.manchester.ac.uk/portal/files/85753741/paper.pdf}{HFK19}] Demystifying Crypto Mining: Performance Analysis and Optimizations of PoW Algorithms. Runchao Han, Nikolaos Foutris, Christos Kotselidis. IEEE International Symposium on Performance Analysis of Systems and Software. \textbf{ISPASS'19, best paper nominee}.
\end{itemize}

\end{rSection}

\begin{rSection}{Talks}
\begin{itemize}
    \item VRF-Based Mining: Simple Non-Outsourceable Cryptocurrency Mining. CBT workshop, virtual. September, 2020.
    \item Demystifying Crypto-Mining: Analysis and Optimizations of Memory-Hard PoW Algorithms. Virtual seminar at Huawei Noah's Ark Lab. July, 2020.
    \item Scaling blockchains via sharding. Confirmation seminar at Monash University. March, 2020.
    \item On the optionality and fairness of Atomic Swaps. AFT conference presentation at Zurich. October, 2019.
\end{itemize}
\end{rSection}


\begin{rSection}{Teaching}
\begin{itemize}
    \item Teaching associate for FIT 5214 Blockchain, Monash University. 2019 Fall.
\end{itemize}
\end{rSection}


\begin{rSection}{Professional services}
\subsection*{External reviewer}
\begin{itemize}
    \item 2020: AFT, ICDCS, SRDS, ACNS, AsiaCCS, TrustCom, ACISP, ICBC, TDSC, The Computer Journal, IEEE IoT Journal, IEEE Software Journal, IEEE Transaction of Service Computing
    \item 2019: Indocrypt, TrustCom, Future Generation Computing System
\end{itemize}
\end{rSection}

\begin{rSection}{Work experience}

\begin{rSubsubsection}{Bytom Blockchain}{September 2018 - January 2019}{Intern Blockchain Engineer}{Hangzhou, China}
\end{rSubsubsection}

\begin{rSubsubsection}{CNIC, Chinese Academy of Sciences}{June 2017 - July 2017}{Intern Researcher}{Beijing, China}
\end{rSubsubsection}

\end{rSection}

\begin{rSection}{Additional Information}

\begin{tabular}{ @{} >{\bfseries}l @{\hspace{6ex}} l }
Github & SebastianElvis \\
Wechat & elvisage \\
References & Available upon request
\end{tabular}

\end{rSection}

\end{document}
